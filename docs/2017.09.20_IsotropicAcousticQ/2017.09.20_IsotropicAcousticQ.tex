\documentclass[10pt,fleqn]{article}
\usepackage{amsmath}
\usepackage{cancel}
%\usepackage{amssymb}
\usepackage[letterpaper, landscape, margin=1in]{geometry}

\setlength{\parindent}{0pt}

\begin{document}

\section{Isotropic acoustic modeling with Q absorbing boundaries, self adjoint equation}
John Washbourne, Ken Bube\\
\today
\vspace{10pt}

Equation \ref{eq:system} shows the acoustic isotropic second order modeling equation with absorbing boundaries implemented using amplitude only Q.
Note that we apply the time derivative in the term $\displaystyle \frac{\omega}{Q} \partial_t p $
using a backward numerical difference.  We have tested a centered difference and found it to be much less stable.

\begin{equation}
\frac{1}{v^2} \left( \partial_t^2 p + \frac{\omega}{Q} \partial_t p \right) = \nabla p + S
\label{eq:system}
\end{equation}

\section{Time update}
\subsection{Time update numerical difference formulas, first and second}
\begin{equation}
\begin{aligned}
\partial_t p  &= \frac{1}{\Delta} \left[ p_{(t)} - p_{(t - \Delta)} \right] \\[10pt]
\partial_t^2 p &= \frac{1}{\Delta^2} \left[ p_{(t+\Delta)} - 2 p_{(t)} + p_{(t - \Delta)} \right] \quad \rightarrow \qquad p_{(t+\Delta)} = \Delta^2 \partial_t^2 p + 2 p_{(t)} - p_{(t - \Delta)}
\end{aligned}
\end{equation}

\subsection{Rearrange equation \ref{eq:system} for $\partial_t^2 p$}
\begin{equation}
\partial_t^2 p = v^2 \left( \nabla p + S \right) - \frac{\omega}{Q} \partial_t p 
\end{equation}

\subsection{Apply time update difference formulas and rearrange}
\begin{equation}
p_{(t+\Delta)} = \Delta^2 v^2 \left( \nabla p + S \right) - \Delta \frac{\omega}{Q} \left[ p_{(t)} - p_{(t - \Delta)} \right] 
+ 2 p_{(t)} - p_{(t - \Delta)} 
\end{equation}

\newpage
\section{Linearization and Born modeling equation}

\subsection{Nonlinear modeling equation}
\begin{equation}
\frac{1}{v^2} \left( \partial_t^2 p + \frac{\omega}{Q} \partial_t p \right) = \nabla p + S
\end{equation}

\subsection{Taylor expand $\displaystyle \frac{1}{v^2} \rightarrow \left( \frac{1}{v_0^2} - \frac{2}{v_0^3} \delta v \right) $
and replace $\displaystyle p \rightarrow \displaystyle (p_0 + \delta p)$}
\begin{equation}
\left( \frac{1}{v_0^2} - \frac{2}{v_0^3} \delta v \right) \left( \partial_t^2 p_0 + \partial_t^2 \delta p +
\frac{\omega}{Q} \partial_t p_0 + \frac{\omega}{Q} \partial_t \delta p \right) =
\nabla p_0 + \nabla \delta p + S
\end{equation}

\subsection{Expand}
\begin{equation}
  \frac{1}{v_0^2} \partial_t^2 p_0 
+ \frac{1}{v_0^2} \partial_t^2 \delta p
+ \frac{1}{v_0^2} \frac{\omega}{Q} \partial_t p_0 
+ \frac{1}{v_0^2} \frac{\omega}{Q} \partial_t \delta p
- \frac{2}{v_0^3} \delta v \partial_t^2 p_0
- \frac{2}{v_0^3} \delta v \partial_t^2 \delta p
- \frac{2}{v_0^3} \delta v \frac{\omega}{Q} \partial_t p_0
- \frac{2}{v_0^3} \delta v \frac{\omega}{Q} \partial_t \delta p
= \nabla p_0 + \nabla \delta p + S 
\end{equation}

\subsection{Cancel reference terms}
\begin{equation}
  \cancel{ \frac{1}{v_0^2} \partial_t^2 p_0 }
+ \frac{1}{v_0^2} \partial_t^2 \delta p
+ \cancel{ \frac{1}{v_0^2} \frac{\omega}{Q} \partial_t p_0 }
+ \frac{1}{v_0^2} \frac{\omega}{Q} \partial_t \delta p
- \frac{2}{v_0^3} \delta v \partial_t^2 p_0
- \frac{2}{v_0^3} \delta v \partial_t^2 \delta p 
- \frac{2}{v_0^3} \delta v \frac{\omega}{Q} \partial_t p_0
- \frac{2}{v_0^3} \delta v \frac{\omega}{Q} \partial_t \delta p 
= \cancel{ \nabla p_0 } + \nabla \delta p + \cancel{ S }
\end{equation}

\subsection{Zero terms second order in perturbations $\delta p,\delta v$ }
\begin{equation}
\frac{1}{v_0^2} \partial_t^2 \delta p
+ \frac{1}{v_0^2} \frac{\omega}{Q} \partial_t \delta p
- \frac{2}{v_0^3} \delta v \partial_t^2 p_0
- \cancel{ \frac{2}{v_0^3} \delta v \partial_t^2 \delta p }
- \frac{2}{v_0^3} \delta v \frac{\omega}{Q} \partial_t p_0
- \cancel{ \frac{2}{v_0^3} \delta v \frac{\omega}{Q} \partial_t \delta p }
= \nabla \delta p 
\end{equation}

\subsection{Rearrange for the Born modeling equation}
\begin{equation}
\begin{aligned}
\frac{1}{v_0^2} \left( \partial_t^2 \delta p + \frac{\omega}{Q} \partial_t \delta p \right) 
&= \nabla \delta p + \frac{2}{v_0^3} \delta v \left( \partial_t^2 p_0 + \frac{\omega}{Q} \partial_t p_0 \right) \\[10pt]
&= \nabla \delta p + \frac{2}{v_0^3} \delta v \left( v_0^2 \nabla p_0 + v_0^2 S \right) \\[10pt]
\end{aligned}
\end{equation}

Note: it is procedurally simplest to serialize the quantity $\displaystyle \left( v_0^2 \nabla p_0 + v_0^2 S \right)$ when we perform 
nonlinear forward modeling, and use that field as the Born source for the linearized forward and adjoint operators.

\end{document}
