\documentclass[10pt,fleqn]{article}
\usepackage{amsmath}
\usepackage{cancel}
\usepackage[letterpaper, landscape, margin=1in]{geometry}

\setlength{\parindent}{0pt}

\begin{document}

\section{Isotropic visco-acoustic variable density second order self-adjoint system}
John Washbourne, Ken Bube\\
September 20, 2013

\section{Introduction}
This note shows the derivation of \textit{time update equations} and the linearization 
for the isotropic
variable density self-adjoint system. We implement attenuation with a monochromatic
approximation to Maxwell bodies, and use this attenuation model to implement zero
outgoing absorbing boundary conditions on the exterior of the modeling domain.
\vspace{10pt}

The time update equations are used to advance solutions in time, expressing the pressure
wavefield at time $p_{(t + \Delta)}$ as a function of $p_{(t - \Delta)}$ and $p_{(t)}$.

\subsection{Symbols}
\begin{center}
\begin{tabular}{ll} \\[-10pt]
$\partial_t$ & $\displaystyle \frac{\partial }{\partial t}$ \\[10pt]
$\nabla p $ & Laplacian: $ \left(
	\displaystyle \frac{\partial }{\partial x} +
	\displaystyle \frac{\partial }{\partial y} +
	\displaystyle \frac{\partial }{\partial z} \right) $ \\[10pt]
$\omega, Q$ & reference frequency for attentuation, attenuation at frequency $\omega$ \\[10pt]
$P$ & Pressure wavefields\\[10pt]
$S(x,y,z,t)$ & Pressure source term\\[10pt]
$b$ & buoyancy = $\displaystyle 1/\rho $ (reciprocal density) \\[10pt]
$ \{\ V_p\ \} $ & Material parameters \\[10pt]
\end{tabular}
\end{center}

\subsection{Modeling system}
Equation \ref{eq:system} shows the modeling system with absorbing boundaries
implemented using amplitude only (dissipation only, no dispersion) Q.
\vspace{10pt}

We apply the time derivative in the term $\displaystyle \frac{\omega}{Q} \partial_t p$
using a backward none-sided numerical difference. We tested both forward one-sided
and centered difference alternatives and found them to be less stable.

\begin{equation}
\frac{1}{v^2} \left( \partial_t^2 p + \frac{\omega}{Q} \partial_t p \right) = \nabla p + S
\label{eq:system}
\end{equation}

\newpage
\section{Time update equations}

\subsection{Time update numerical difference formulas, first and second order}
\begin{equation}
\partial_t p = \frac{1}{\Delta} \left[ p_{(t)} - p_{(t - \Delta)} \right]
\label{eq:diff1}
\end{equation}

\begin{equation}
\begin{aligned}
\partial_t^2 p &= \frac{1}{\Delta^2} \left[ p_{(t+\Delta)} - 2 p_{(t)} + p_{(t - \Delta)} \right] \\[10pt]
p_{(t+\Delta)} &= \Delta^2 \partial_t^2 p + 2 p_{(t)} - p_{(t - \Delta)}
\end{aligned}
\label{eq:diff2}
\end{equation}

\subsection{Rearrange equation \ref{eq:system} for $\partial_t^2 p$}
\begin{equation}
\partial_t^2 p = v^2 \left( \nabla p + S \right) - \frac{\omega}{Q} \partial_t p 
\end{equation}

\subsection{Apply equations \ref{eq:diff1} and \ref{eq:diff2}, and rearrange}
\begin{equation}
p_{(t+\Delta)} = \Delta^2 v^2 \left( \nabla p + S \right) - \Delta \frac{\omega}{Q} \left[ p_{(t)} - p_{(t - \Delta)} \right] 
+ 2 p_{(t)} - p_{(t - \Delta)} 
\end{equation}

\newpage
\section{Linearization and Born modeling equation}

\subsection{Nonlinear modeling equation}
\begin{equation}
\frac{1}{v^2} \left( \partial_t^2 p + \frac{\omega}{Q} \partial_t p \right) = \nabla p + S
\end{equation}

\subsection{Taylor expand $\displaystyle \frac{1}{v^2} \rightarrow \left( \frac{1}{v_0^2} - \frac{2}{v_0^3} \delta v \right) $
and replace $\displaystyle p \rightarrow \displaystyle (p_0 + \delta p)$}
\begin{equation}
\left( \frac{1}{v_0^2} - \frac{2}{v_0^3} \delta v \right) \left( \partial_t^2 p_0 + \partial_t^2 \delta p +
\frac{\omega}{Q} \partial_t p_0 + \frac{\omega}{Q} \partial_t \delta p \right) =
\nabla p_0 + \nabla \delta p + S
\end{equation}

\subsection{Expand}
\begin{equation}
  \frac{1}{v_0^2} \partial_t^2 p_0 
+ \frac{1}{v_0^2} \partial_t^2 \delta p
+ \frac{1}{v_0^2} \frac{\omega}{Q} \partial_t p_0 
+ \frac{1}{v_0^2} \frac{\omega}{Q} \partial_t \delta p
- \frac{2}{v_0^3} \delta v \partial_t^2 p_0
- \frac{2}{v_0^3} \delta v \partial_t^2 \delta p
- \frac{2}{v_0^3} \delta v \frac{\omega}{Q} \partial_t p_0
- \frac{2}{v_0^3} \delta v \frac{\omega}{Q} \partial_t \delta p
= \nabla p_0 + \nabla \delta p + S 
\end{equation}

\subsection{Cancel reference terms}
\begin{equation}
  \cancel{ \frac{1}{v_0^2} \partial_t^2 p_0 }
+ \frac{1}{v_0^2} \partial_t^2 \delta p
+ \cancel{ \frac{1}{v_0^2} \frac{\omega}{Q} \partial_t p_0 }
+ \frac{1}{v_0^2} \frac{\omega}{Q} \partial_t \delta p
- \frac{2}{v_0^3} \delta v \partial_t^2 p_0
- \frac{2}{v_0^3} \delta v \partial_t^2 \delta p 
- \frac{2}{v_0^3} \delta v \frac{\omega}{Q} \partial_t p_0
- \frac{2}{v_0^3} \delta v \frac{\omega}{Q} \partial_t \delta p 
= \cancel{ \nabla p_0 } + \nabla \delta p + \cancel{ S }
\end{equation}

\subsection{Zero terms second order in perturbations $\delta p,\delta v$ }
\begin{equation}
\frac{1}{v_0^2} \partial_t^2 \delta p
+ \frac{1}{v_0^2} \frac{\omega}{Q} \partial_t \delta p
- \frac{2}{v_0^3} \delta v \partial_t^2 p_0
- \cancel{ \frac{2}{v_0^3} \delta v \partial_t^2 \delta p }
- \frac{2}{v_0^3} \delta v \frac{\omega}{Q} \partial_t p_0
- \cancel{ \frac{2}{v_0^3} \delta v \frac{\omega}{Q} \partial_t \delta p }
= \nabla \delta p 
\end{equation}

\subsection{Rearrange for the Born modeling equation}
\begin{equation}
\begin{aligned}
\frac{1}{v_0^2} \left( \partial_t^2 \delta p + \frac{\omega}{Q} \partial_t \delta p \right) 
&= \nabla \delta p + \frac{2}{v_0^3} \delta v \left( \partial_t^2 p_0 + \frac{\omega}{Q} \partial_t p_0 \right) \\[10pt]
&= \nabla \delta p + \frac{2}{v_0^3} \delta v \left( v_0^2 \nabla p_0 + v_0^2 S \right) \\[10pt]
\end{aligned}
\end{equation}

Note: it is procedurally simplest to serialize the quantity
$\displaystyle \left( v_0^2 \nabla p_0 + v_0^2 S \right)$ when we perform nonlinear
forward modeling, and use that field as the Born source for the linearized forward
and adjoint operators.  
\end{document}
